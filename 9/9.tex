\documentclass[a4paper]{article}
\usepackage[pdftex]{hyperref}
\usepackage[latin1]{inputenc}
\usepackage[english]{babel}
\usepackage{a4wide}
\usepackage{amsmath}
\usepackage{amssymb}
\usepackage{algorithmic}
\usepackage{algorithm}
\usepackage{ifthen}
\usepackage{listings}
\usepackage{hyperref}
\usepackage{minted}
% move the asterisk at the right position
\lstset{basicstyle=\ttfamily,tabsize=4,literate={*}{${}^*{}$}1}
%\lstset{language=C,basicstyle=\ttfamily}
%\usepackage{moreverb}
\usepackage{palatino}
\usepackage{multicol}
\usepackage{tabularx}
%\usepackage{comment}
\usepackage{verbatim}
\usepackage{color}

%% pdflatex?
\newif\ifpdf
\ifx\pdfoutput\undefined
\pdffalse % we are not running PDFLaTeX
\else
\pdfoutput=1 % we are running PDFLaTeX
\pdftrue
\fi
\ifpdf
\usepackage[pdftex]{graphicx}
\else
\usepackage{graphicx}
\fi
\ifpdf
\DeclareGraphicsExtensions{.pdf, .jpg}
\else
\DeclareGraphicsExtensions{.eps, .jpg}
\fi

\parindent=0cm
\parskip=0cm

\setlength{\columnseprule}{0.4pt}
\addtolength{\columnsep}{2pt}

\addtolength{\textheight}{5.5cm}
\addtolength{\topmargin}{-26mm}
\pagestyle{empty}

%%
%% Sheet setup
%% 
\newcommand{\coursename}{Computer Architecture and Programming Languages}
\newcommand{\courseno}{CO20-320241}
 
\newcommand{\sheettitle}{Homework}
\newcommand{\mytitle}{}
\newcommand{\mytoday}{\textcolor{blue}{November 18th}, 2019}

% Current Assignment number
\newcounter{assignmentno}
\setcounter{assignmentno}{9}

% Current Problem number, should always start at 1
\newcounter{problemno}
\setcounter{problemno}{1}

%%
%% problem and bonus environment
%%
\newcounter{probcalc}
\newcommand{\problem}[2]{
  \pagebreak[2]
  \setcounter{probcalc}{#2}
  ~\\
  {\large \textbf{Problem \textcolor{blue}{\arabic{assignmentno}}.\textcolor{blue}{\arabic{problemno}}} \hspace{0.2cm}\textit{#1}} \refstepcounter{problemno}\vspace{2pt}\\}

\newcommand{\bonus}[2]{
  \pagebreak[2]
  \setcounter{probcalc}{#2}
  ~\\
  {\large \textbf{Bonus Problem \textcolor{blue}{\arabic{assignmentno}}.\textcolor{blue}{\arabic{problemno}}} \hspace{0.2cm}\textit{#1}} \refstepcounter{problemno}\vspace{2pt}\\}

%% some counters  
\newcommand{\assignment}{\arabic{assignmentno}}

%% solution  
\newcommand{\solution}{\pagebreak[2]{\bf Solution:}\\}

%% Hyperref Setup
\hypersetup{pdftitle={Homework \assignment},
  pdfsubject={\coursename},
  pdfauthor={},
  pdfcreator={},
  pdfkeywords={Computer Architecture and Programming Languages},
  %  pdfpagemode={FullScreen},
  %colorlinks=true,
  %bookmarks=true,
  %hyperindex=true,
  bookmarksopen=false,
  bookmarksnumbered=true,
  breaklinks=true,
  %urlcolor=darkblue
  urlbordercolor={0 0 0.7}
}

\begin{document}
\coursename \hfill Course: \courseno\\
Jacobs University Bremen \hfill \mytoday\\
\textcolor{blue}{Arsenij Percov}\hfill
\vspace*{0.3cm}\\
\begin{center}
{\Large \sheettitle{} \textcolor{blue}{\assignment}\\}
\end{center}

\problem{}{0}
\solution
a) Jump and Branch control signals feed the mutexes at the end of the single cycle data path and output of them goes to PC as input. By default the value would be incremented by 4. These are the only two needed scenarios, therefore the explicit write signal is pointless.\\\\
b) For multi-cycle data path, we need to specify which path to pick in determining the address for next instructions.
\\
\url{https://stackoverflow.com/questions/40835418/why-an-explicit-single-cycle-datapath-is-not-needed#:~:targetText=Therefore%20there%20is%20no%20need,address%20of%20the%20next%20instruction}\\\\


\problem{}{0}
\solution
\\a)
\includegraphics[scale=0.5]{add.png}\\
\includegraphics[scale=0.5]{load.png}\\\\
\begin{tabular}{|c|c|c|c|c|c|c|c|c|}
\hline
Instruction&RegDst&ALUSrc&MemtoReg&Reg Write&Mem Read&Mem Write&Branch&ALUOp  \\ \hline
 \textit{add}&1&0&0&1&0&0&1&0  \\ \hline
        \textit{lw}&0&1&1&1&1&0&0&0 \\ \hline
\end{tabular}\\
\\b) ALU adds when it recieves signal 0010. \\That happens for LW and SW instructions. We add to calculate the addresses. \\
R-type instruction also require ALU to add, if we want to add two registers.
\end{document} 