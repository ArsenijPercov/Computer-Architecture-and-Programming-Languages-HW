\documentclass[a4paper]{article}
\usepackage[pdftex]{hyperref}
\usepackage[latin1]{inputenc}
\usepackage[english]{babel}
\usepackage{a4wide}
\usepackage{amsmath}
\usepackage{amssymb}
\usepackage{algorithmic}
\usepackage{algorithm}
\usepackage{ifthen}
\usepackage{enumitem}
\usepackage{listings}
% move the asterisk at the right position
\lstset{basicstyle=\ttfamily,tabsize=4,literate={*}{${}^*{}$}1}
%\lstset{language=C,basicstyle=\ttfamily}
\usepackage{moreverb}
\usepackage{palatino}
\usepackage{multicol}
\usepackage{tabularx}
\usepackage{comment}
\usepackage{verbatim}
\usepackage{color}
\usepackage{tikz}
\usetikzlibrary{arrows,shapes.gates.logic.US,shapes.gates.logic.IEC,calc}
%% pdflatex?
\newif\ifpdf
\ifx\pdfoutput\undefined
\pdffalse % we are not running PDFLaTeX
\else
\pdfoutput=1 % we are running PDFLaTeX
\pdftrue
\fi

\ifpdf
\DeclareGraphicsExtensions{.pdf, .jpg}
\else
\DeclareGraphicsExtensions{.eps, .jpg}
\fi

\parindent=0cm
\parskip=0cm

\setlength{\columnseprule}{0.4pt}
\addtolength{\columnsep}{2pt}

\addtolength{\textheight}{5.5cm}
\addtolength{\topmargin}{-26mm}
\pagestyle{empty}

%%
%% Sheet setup
%% 
\newcommand{\coursename}{Computer Architecture and Programming Languages}
\newcommand{\courseno}{CO20-320241}
 
\newcommand{\sheettitle}{Homework}
\newcommand{\mytitle}{}
\newcommand{\mytoday}{\textcolor{blue}{October 14th}, 2019}

% Current Assignment number
\newcounter{assignmentno}
\setcounter{assignmentno}{5}

% Current Problem number, should always start at 1
\newcounter{problemno}
\setcounter{problemno}{1}

%%
%% problem and bonus environment
%%
\newcounter{probcalc}
\newcommand{\problem}[2]{
  \pagebreak[2]
  \setcounter{probcalc}{#2}
  ~\\
  {\large \textbf{Problem \textcolor{blue}{\arabic{assignmentno}}.\textcolor{blue}{\arabic{problemno}}} \hspace{0.2cm}\textit{#1}} \refstepcounter{problemno}\vspace{2pt}\\}

\newcommand{\bonus}[2]{
  \pagebreak[2]
  \setcounter{probcalc}{#2}
  ~\\
  {\large \textbf{Bonus Problem \textcolor{blue}{\arabic{assignmentno}}.\textcolor{blue}{\arabic{problemno}}} \hspace{0.2cm}\textit{#1}} \refstepcounter{problemno}\vspace{2pt}\\}

%% some counters  
\newcommand{\assignment}{\arabic{assignmentno}}

%% solution  
\newcommand{\solution}{\pagebreak[2]{\bf Solution:}\\}

%% Hyperref Setup
\hypersetup{pdftitle={Homework \assignment},
  pdfsubject={\coursename},
  pdfauthor={},
  pdfcreator={},
  pdfkeywords={Computer Architecture and Programming Languages},
  %  pdfpagemode={FullScreen},
  %colorlinks=true,
  %bookmarks=true,
  %hyperindex=true,
  bookmarksopen=false,
  bookmarksnumbered=true,
  breaklinks=true,
  %urlcolor=darkblue
  urlbordercolor={0 0 0.7}
}

\begin{document}
\coursename \hfill Course: \courseno\\
Jacobs University Bremen \hfill \mytoday\\
\textcolor{blue}{Arsenij Percov}\hfill
\vspace*{0.3cm}\\
\begin{center}
{\Large \sheettitle{} \textcolor{blue}{\assignment}\\}
\end{center}
\problem{}{0}
\solution
\textcolor{blue}{}
\begin{enumerate}[label=(\alph*)]
\item 
$14_{10} + 37_{10} = 110_2 + 100101_2 = 101011_2$ Carry 1 bit on 3rd position from end.\\
\item
$12_{10} - 27_{10}$\\
$-27_{10}$ Sign is 1, convert 27 to binary, invert, and add 1\\
$27_{10}= 11011_2$
\\
Make it 8 bit\\
00011011, invert\\
11100100, add 1\\
$-27_{10}= 11100101_2$\\
$12_{10}- 27_{10} = 1100_2 + 11100101_2 = 11110001_2$ Invert again, and add 1\\
$00001110_2 + 1 $=\\
$1111_2 = -15 $is the answer.\\
\item 
69 = 0110 1001\\
58 = 0101 1000\\
\begin{tabular}{ccccc}
 &&$0110$&$1001$&  \\
+&&$0101$&$1000$& \\ \hline
=&&$1011$&$10001$&Add 6 to right sum \\
+&&$0001$&$0110$&Add carry and 6\\ \hline
=&&$1100$&$0111$&Add 6 to left sum(bigger than 9)\\
+&&$0110$&$0000$&\\ \hline
=&&$10010$&$0111$&\\\hline
=&$0001$&$0010$&$0111$&\\\hline
\end{tabular}
\item
275 = 0010 0111 0101\\
642 = 0110 0100 0010\\
\begin{tabular}{ccccc}
 &$0010$&$0111$&$0101$&  \\
+&$0110$&$0100$&$0010$& \\ \hline
=&$1000$&$1011$&$0111$&Add 6 to the middle sum \\
+&$0000$&$0110$&$0000$&\\ \hline
=&$1000$&$10001$&$0111$& Carry the bit\\ 
+&$0001$&$0000$&$0000$&\\ \hline
=&$1001$&$0001$&$0111$&\\\hline
\end{tabular}
\item 
6AF + 23C = \\
\begin{tabular}{c|c|c|c|c|}
                \hline
                 &6&A&F&  \\ 
                +&2&3&C&  \\ \hline
                =&8&D&1B&Carry last bit  \\ 
                +&0&1&0&\\ \hline
                =&8&E&B
            \end{tabular}
\item
594 - 3A8 =\\First, find 15 compliment of the second number.\\
$3A8 \xrightarrow{\text{invert}} C57$ 
\begin{tabular}{c|c|c|c|c|}
                \hline
                 &5&9&4&  \\ 
                +&C&5&7&  \\ \hline
                =&1&E&B&Carry bit to least significant digit  \\ 
                =&1&E&C&
            \end{tabular}
\end{enumerate}

 \problem{}{0}
\solution
\textcolor{blue}{}
\begin{enumerate}[label=(\alph*)]
\item $a = b + c$ \\
$ \$t0 = \$s0 + \$s1$\\
MIPS:\\ 
add \$t0,  \$s0,  \$s1   Add b and c, and store it in a\\\\
\item  sub \$t0,  \$s0,  \$s2 Subtract b -d and store it in a\\
add \$t0,  \$t0,  \$s1 Add a to c and store the result in a\\\\
\item  add \$t0,  \$s0, \$s0  Add b and b and store it in a\\
add \$t0,  \$t0,  \$s0 Add a and b and store it in a\\
\\
\item add \$t0, 1, \$s0 Add 1 to b, and store in a\\
add \$t0 \$t0 \$t0 Add a to a, and store in a (multiply by 2)\\
\end{enumerate}

 \problem{}{0}
\solution
\textcolor{blue}{}

Let's first transform register \# of variables a, b, c to binary numbers.\\
a = \$t0 = 8 = $01000_2$\\
b = \$s0 = 16 = $10000_2$\\
c = \$s1 = 17 = $10001_2$\\
d = \$s2 = 18 = $10010_2$\\
\begin{enumerate}[label=(\alph*)]
\item \begin{tabular}{|c|c|c|c|c|c|c|}
\hline
MIPS instruction&op&rs&rt&rd&shamt&funct\\ \hline
add \$t0 \$s0 \$s1 &000000&10000&10001&01000&00000&100000\\ \hline
\end{tabular}

\item \begin{tabular}{|c|c|c|c|c|c|c|}
\hline
MIPS instruction&op&rs&rt&rd&shamt&funct\\ \hline
sub \$t0,  \$s0,  \$s2 &000000&10000&10010&01000&00000&100010\\ \hline
add \$t0,  \$t0,  \$s1 &000000&01000&10001&01000&00000&100000\\ \hline
\end{tabular}


\end{enumerate}
\problem{}{0}
\solution
\textcolor{blue}{}
B[5] = A[4] + A[2]\\
lw \$t0, 16(\$s0) Load A[4] into temp. reg. \$t0 \\
lw \$t1, 8(\$s0)  Load A[2] into temp. reg \$t1 \\
lw \$t2, 20(\$s1) Load B[5] into temp. reg \$t2 \\
add \$t2, \$t0, \$t1 Add A[4] and A[2] and store it to \$t2\\
sw \$t2, 20(\$s1) Save the result of \$t2 to B[5]

\problem{}{0}
\solution
\textcolor{blue}{}
add \$t1, \$t0 , 7  Add 7 to x, and store it at \$t1\\
add \$t1, \$t1, \$t1 Multiply \$t1 by 2.\\
add \$t1, \$t1, \$t1 Multiply \$t1 by 2 (Now we got offset for A[x+7].\\
add \$t1, \$t1, \$s0 Get adress of A[x+7]\\
add \$t2, \$t0 , 2  Add 2 to x, and store it at \$t2\\
add \$t2, \$t2, \$t2 Multiply \$t2 by 2.\\
add \$t2, \$t2, \$t2 Multiply \$t2 by 2 (Now we got offset for A[x+2].\\
add \$t2, \$t2, \$s0 Get adress of A[x+2]\\
add \$t0, \$t0, \$t0 Multiply \$t0 by 2\\
add \$t0, \$t0, \$t0 Multiply \$t0 by 2\\
add \$t0, \$t0, \$s1 Add offset for B[x]\\
lw \$t3, 0(\$t1) Load value of A[x+7]\\
lw \$t4, 0(\$t2) Load value for A [x+2]\\
add \$t5, \$t3, \$t4. Add value of A[x+7] to A[x+2] and store it at \$t5\\
sw \$t5, 0(\$t0) Store value of sum to B[x] (\$t5)

 
\problem{}{0}
\solution
\textcolor{blue}{}
It will change number of bits used for representing the id of registers. Before it was  5, but now since we have only 16 numbers, we can use 4, since it is enough to represent all of them. Since every MIPS instruction have 32 bits, we will have 2 free bits(1 from first register, one from second), that can be used for representing bigger constant (18 bits instead of 16)

\end{document}