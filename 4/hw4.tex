\documentclass[a4paper]{article}
\usepackage[pdftex]{hyperref}
\usepackage[latin1]{inputenc}
\usepackage[english]{babel}
\usepackage{a4wide}
\usepackage{amsmath}
\usepackage{amssymb}
\usepackage{algorithmic}
\usepackage{algorithm}
\usepackage{ifthen}
\usepackage{listings}
% move the asterisk at the right position
\lstset{basicstyle=\ttfamily,tabsize=4,literate={*}{${}^*{}$}1}
%\lstset{language=C,basicstyle=\ttfamily}
\usepackage{moreverb}
\usepackage{palatino}
\usepackage{multicol}
\usepackage{tabularx}
\usepackage{comment}
\usepackage{verbatim}
\usepackage{color}
\usepackage{tikz}
\usetikzlibrary{arrows,shapes.gates.logic.US,shapes.gates.logic.IEC,calc}
%% pdflatex?
\newif\ifpdf
\ifx\pdfoutput\undefined
\pdffalse % we are not running PDFLaTeX
\else
\pdfoutput=1 % we are running PDFLaTeX
\pdftrue
\fi

\ifpdf
\DeclareGraphicsExtensions{.pdf, .jpg}
\else
\DeclareGraphicsExtensions{.eps, .jpg}
\fi

\parindent=0cm
\parskip=0cm

\setlength{\columnseprule}{0.4pt}
\addtolength{\columnsep}{2pt}

\addtolength{\textheight}{5.5cm}
\addtolength{\topmargin}{-26mm}
\pagestyle{empty}

%%
%% Sheet setup
%% 
\newcommand{\coursename}{Computer Architecture and Programming Languages}
\newcommand{\courseno}{CO20-320241}
 
\newcommand{\sheettitle}{Homework}
\newcommand{\mytitle}{}
\newcommand{\mytoday}{\textcolor{blue}{October 7th}, 2019}

% Current Assignment number
\newcounter{assignmentno}
\setcounter{assignmentno}{4}

% Current Problem number, should always start at 1
\newcounter{problemno}
\setcounter{problemno}{1}

%%
%% problem and bonus environment
%%
\newcounter{probcalc}
\newcommand{\problem}[2]{
  \pagebreak[2]
  \setcounter{probcalc}{#2}
  ~\\
  {\large \textbf{Problem \textcolor{blue}{\arabic{assignmentno}}.\textcolor{blue}{\arabic{problemno}}} \hspace{0.2cm}\textit{#1}} \refstepcounter{problemno}\vspace{2pt}\\}

\newcommand{\bonus}[2]{
  \pagebreak[2]
  \setcounter{probcalc}{#2}
  ~\\
  {\large \textbf{Bonus Problem \textcolor{blue}{\arabic{assignmentno}}.\textcolor{blue}{\arabic{problemno}}} \hspace{0.2cm}\textit{#1}} \refstepcounter{problemno}\vspace{2pt}\\}

%% some counters  
\newcommand{\assignment}{\arabic{assignmentno}}

%% solution  
\newcommand{\solution}{\pagebreak[2]{\bf Solution:}\\}

%% Hyperref Setup
\hypersetup{pdftitle={Homework \assignment},
  pdfsubject={\coursename},
  pdfauthor={},
  pdfcreator={},
  pdfkeywords={Computer Architecture and Programming Languages},
  %  pdfpagemode={FullScreen},
  %colorlinks=true,
  %bookmarks=true,
  %hyperindex=true,
  bookmarksopen=false,
  bookmarksnumbered=true,
  breaklinks=true,
  %urlcolor=darkblue
  urlbordercolor={0 0 0.7}
}

\begin{document}
\coursename \hfill Course: \courseno\\
Jacobs University Bremen \hfill \mytoday\\
\textcolor{blue}{Arsenij Percov}\hfill
\vspace*{0.3cm}\\
\begin{center}
{\Large \sheettitle{} \textcolor{blue}{\assignment}\\}
\end{center}

 \problem{}{0}
\solution
\textcolor{blue}{}
Let's first transform the logic circuit to logical expression:
\\$(A \oplus B) \cdot \overline{B\oplus C} \cdot C$
\\Now let's make a truth table for this expression:\\
\begin{center}
	\begin{tabular}{|c|c|c|c|}
	\hline
	A&B&C&X \\
	\hline
	0&0&0&0 \\
	\hline
	1&0&0&0 \\
	\hline
	0&1&0&0 \\
	\hline
	0&0&1&0 \\
	\hline
	1&1&0&0 \\
	\hline
	1&0&1&0 \\
	\hline
	0&1&1&1 \\
	\hline
	1&1&1&0	\\	
	\hline		
	\end{tabular}
\end{center}
From the truth table we see that the combination of A = 0, and B = C = 1 is the only combination leading to X be 1.\\\\
\problem{}{0}
\solution
\textcolor{blue}{}
Let's start with writing the equation:
\\
$\overline{(A \cdot B) + C} \oplus (\overline{A}\cdot (B+C))$\\
Truth table:
\\
\begin{center}
	\begin{tabular}{|c|c|c|c|}
	\hline
	A&B&C&X \\
	\hline
	0&0&0&1 \\
	\hline
	1&0&0&1 \\
	\hline
	0&1&0&0 \\
	\hline
	0&0&1&1 \\
	\hline
	1&1&0&0 \\
	\hline
	1&0&1&0 \\
	\hline
	0&1&1&1 \\
	\hline
	1&1&1&0	\\	
	\hline		
	\end{tabular}
\end{center}
And the sum of products is :
\\$(\overline{A} \cdot \overline{B} \cdot \overline{C})+(A \cdot \overline{B} \cdot \overline{C})+(\overline{A} \cdot \overline{B} \cdot C)+(\overline{A} \cdot B \cdot C)$\\
\problem{}{0}
\solution
\textcolor{blue}{}
a) Convert 27 to binary number:\\
27/2 = 13 (1)\\
13/2 = 6 (1)\\
6/2 = 3 (0)\\
3/2 = 1 (1)\\
1/2 = 0 (1)\\
The binary representation is 11011. Since it is positive number, the 2's complement of 8 bit is (sign is +, so first digit is 0) 00011011\\\\
b) 66 to binary:\\
66/2 = 33 (0)\\
33/2 = 16 (1)\\
16/2 = 8 (0)\\
8/2 = 4 (0)\\
4/2 = 2 (0)\\
2/2 = 1 (0)\\
1/2 = 0 (1)\\
The sign is +, so first digit is 0. Binary: 1000010. 2's complement 01000010.\\\\
c) -18\\
Convert 18 to binary:\\
18/2 = 9 (0)\\
9/2 = 4 (1)\\
4/2 = 2 (0)\\
2/2 = 1 (0)\\
1/2 = 0 (1)\\
Binary: 100010. The sign is -, so we need to invert the binary representation and add 1. \\
Invert:\\
$00100010 \xrightarrow 00011101 $\\
Add 1:\\
00011101 + 1 = 00011110\\
Make it 8 bit, first digit is 1.\\
2's complement: 10011110\\\\
d) 127 :
\\Positive number, so only convert to binary:\\
127/2 = 63 (1)\\
63/2 = 31 (1)\\
31/2 = 15 (1)\\
15/2 = 7 (1) \\
7/2 = 3 (1) \\
3/2 = 1 (1) \\
1/2 = 0 (1)\\
Binary: 1111111. 2's complement : 01111111 \\\\
e)-127. Convert to binary, invert, add one. We know that 127 is 1111111 from previous problem.\\
Invert:\\
10000000
Add one:\\
10000001\\
Negative, so first digit is 1. 
\\ 2's complement : 10000001\\\\
f)-128. Convert to binary, invert, add one.
We can find binary representation by adding one to 127, which is 01111111 + 1 = 10000000. \\
Invert: \\
01111111\\
Add one:\\
10000000\\
2's complement: 10000000\\\\
g)131\\
Binary:\\
131/2 = 65 (1)\\
65/2 = 32 (1)\\
32/2 = 16 (0)\\
16/2 = 8 (0)\\
8/2 = 4 (0)\\
4/2 = 2 (0)\\
2/2 = 1 (0)\\
1/2 = 0 (1)\\
Binary: 10000011. \\
It is out of the range, so we cannot represent it in 8-bit 2's complement.\\\\
h) -7. \\
Convert 7 to binary, invert, and add 1.\\
Convert:\\
7/2 = 3 (1)\\
3/2 = 1 (1)\\
1/2 = 0 (1)\\
Binary:\\
00000111
Invert:\\
11111000\\
Add one:\\
11111001\\
First digit is 1,2's complement 11111001\\
\problem{}{0}
\solution
\textcolor{blue}{}
a) 00011000 :\\
First digit is 0, number is positive. Convert to decimal:\\
$2^4 + 2^3 = 16 + 8 = 24$\\
\\b) 11110101\\
Negative number:\\
Subtract one, invert, convert to decimal, add minus.
\\
11110101 - 1 = 11110100\\
Invert:\\
00001011\\
Convert to decimal:\\
$2^3+2^1+2^0 = 8+2+1 = 11$ \\
Decimal: -11\\
c) 01011011\\
Convert:\\\\
$01011011 = 2^6 + 2^4 + 2^3 + 2^1 + 2^0 = 64 + 16 + 8 + 2 + 1 = 91$\\
d)10110110\\
Negative number:\\
Subtract\\
10110110 - 1 = 10110101\\
Invert:\\
01001010\\
Convert:\\
$01001010 = 2^6 + 2^3 + 2^1 = 64 + 8 + 2 = 74$\\
Negate:
-74\\\\
e) 11111111\\
Negative number:\\
Subtract:\\
11111111 - 1 = 11111110\\
Invert:\\
00000001\\
Convert:\\
$1 = 2^0 = 1$
\\Negate: \\
-1\\
f)01101111\\
Positive number:\\
Convert:\\
$01101111 = 2^6 +2^5 + 2^3 +2^2 + 2^1 + 2^0 = 64 + 32 + 8 + 4 + 2 + 1 = 111$\\\\
g) 10000001\\
Negative number:\\
Subtract:\\
10000001 - 1 = 10000000\\
Invert:\\
01111111\\
Convert:\
$01111111 = 2^6 +2^5 + 2^4 + 2^3 +2^2 + 2^1 + 2^0 = 64 + 32 + 16 + 8 + 4 + 2 + 1 = 127$\\
Negate:\\
-127\\\\
h) 10000000\\
Negative number:\\
Subtract:\\
10000000 - 1 = 01111111\\
Invert:\\
10000000\\
Convert:
$10000000 = 2^7 = 128$\\
Negate:\\
-128\\
\problem{}{0}
\solution
\textcolor{blue}{}
a) 27 + 36 \\
Convert each digit to binary numbers:\\
27 = 0010 0111\\
36 = 0011 0110\\

\begin{tabular}{cccc}
 &$0010$&$0111$&  \\
+&$0011$&$0110$& \\ \hline
=&$0101$&$1101$&Add 6 to right sum \\
+&$0000$&$0110$&\\ \hline
=&$0110$&$0011$
\end{tabular}\\ \\
b) 73 + 29 \\
73 = 0111 0011\\
29 = 0010 1001\\

\begin{tabular}{cccc}
 &$0111$&$0011$&  \\
+&$0010$&$1001$& \\ \hline
 &$1001$&$1100$&More than nine, add 6 to last digit\\
+&$0000$&$0110$&\\ \hline
 &$1010$&$0010$&More than nine, add 6 to first digit\\
+&$0110$&$0010$&\\ \hline
=&$0001$ $0000$&$0010$&\\ 
\end{tabular}\\ \\
\problem{}{0}
\solution
\textcolor{blue}{}
a) 00000000 is smallest number and it is equal to 0. Biggest number would be 11111111. Convert to decimal:
\\$11111111_2 = 2^7 + 2^6 + 2^5 + 2^4 + 2^3 + 2^2 + 2^1 + 2^0 = 128 + 64 +32 +16 +8 +4 +2 +1 = 256$. Range 0-255.\\\\
b) 1 bit for sign, and 7 bits left, so we have $2^7$ numbers. If it is 2's compliment, then we have one more negative number, than positive. Therefore range is $[-2^7, 2^7-1]$\\\\
c)Same as a, but  $[0, 2^{11}]$\\\\
d)Same as b, but  $[-2^{10}, 2^{10}-1]$\\\\
e) Same as b, but  $[-2^{15}, 2^{15}-1]$\\\\
\end{document}