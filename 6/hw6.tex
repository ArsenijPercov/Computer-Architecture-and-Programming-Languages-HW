\documentclass[a4paper]{article}
\usepackage[pdftex]{hyperref}
\usepackage[latin1]{inputenc}
\usepackage[english]{babel}
\usepackage{a4wide}
\usepackage{amsmath}
\usepackage{amssymb}
\usepackage{algorithmic}
\usepackage{algorithm}
\usepackage{ifthen}
\usepackage{enumitem}
\usepackage{listings}
% move the asterisk at the right position
\lstset{basicstyle=\ttfamily,tabsize=4,literate={*}{${}^*{}$}1}
%\lstset{language=C,basicstyle=\ttfamily}
\usepackage{moreverb}
\usepackage{palatino}
\usepackage{multicol}
\usepackage{tabularx}
\usepackage{comment}
\usepackage{verbatim}
\usepackage{color}
\usepackage{tikz}
\usetikzlibrary{arrows,shapes.gates.logic.US,shapes.gates.logic.IEC,calc}
%% pdflatex?
\newif\ifpdf
\ifx\pdfoutput\undefined
\pdffalse % we are not running PDFLaTeX
\else
\pdfoutput=1 % we are running PDFLaTeX
\pdftrue
\fi

\ifpdf
\DeclareGraphicsExtensions{.pdf, .jpg}
\else
\DeclareGraphicsExtensions{.eps, .jpg}
\fi

\parindent=0cm
\parskip=0cm

\setlength{\columnseprule}{0.4pt}
\addtolength{\columnsep}{2pt}

\addtolength{\textheight}{5.5cm}
\addtolength{\topmargin}{-26mm}
\pagestyle{empty}

%%
%% Sheet setup
%% 
\newcommand{\coursename}{Computer Architecture and Programming Languages}
\newcommand{\courseno}{CO20-320241}
 
\newcommand{\sheettitle}{Homework}
\newcommand{\mytitle}{}
\newcommand{\mytoday}{\textcolor{blue}{October 21th}, 2019}

% Current Assignment number
\newcounter{assignmentno}
\setcounter{assignmentno}{6}

% Current Problem number, should always start at 1
\newcounter{problemno}
\setcounter{problemno}{1}

%%
%% problem and bonus environment
%%
\newcounter{probcalc}
\newcommand{\problem}[2]{
  \pagebreak[2]
  \setcounter{probcalc}{#2}
  ~\\
  {\large \textbf{Problem \textcolor{blue}{\arabic{assignmentno}}.\textcolor{blue}{\arabic{problemno}}} \hspace{0.2cm}\textit{#1}} \refstepcounter{problemno}\vspace{2pt}\\}

\newcommand{\bonus}[2]{
  \pagebreak[2]
  \setcounter{probcalc}{#2}
  ~\\
  {\large \textbf{Bonus Problem \textcolor{blue}{\arabic{assignmentno}}.\textcolor{blue}{\arabic{problemno}}} \hspace{0.2cm}\textit{#1}} \refstepcounter{problemno}\vspace{2pt}\\}

%% some counters  
\newcommand{\assignment}{\arabic{assignmentno}}

%% solution  
\newcommand{\solution}{\pagebreak[2]{\bf Solution:}\\}

%% Hyperref Setup
\hypersetup{pdftitle={Homework \assignment},
  pdfsubject={\coursename},
  pdfauthor={},
  pdfcreator={},
  pdfkeywords={Computer Architecture and Programming Languages},
  %  pdfpagemode={FullScreen},
  %colorlinks=true,
  %bookmarks=true,
  %hyperindex=true,
  bookmarksopen=false,
  bookmarksnumbered=true,
  breaklinks=true,
  %urlcolor=darkblue
  urlbordercolor={0 0 0.7}
}

\begin{document}
\coursename \hfill Course: \courseno\\
Jacobs University Bremen \hfill \mytoday\\
\textcolor{blue}{Arsenij Percov}\hfill
\vspace*{0.3cm}\\
\begin{center}
{\Large \sheettitle{} \textcolor{blue}{\assignment}\\}
\end{center}
\problem{}{0}
\solution
\textcolor{blue}{}
Biggest reason for that is the fact that we have 32 registers on most architectures. In order to represent all 32 possible options we need exactly 5 bits, since 11111 represents 31, and 00000 represents 0.\\\\

\problem{}{0}
\solution
\textcolor{blue}{}
a) op = 0, rs = 8, rt = 9, rd = 10, shamt = 0, funct = 34 \\
sub \$t2 \$t0 \$t1\\\\
b) op = 0x23, rs = 17, rt = 18, const = 0x4\\
op = $35_10$\\
lw \$s2 4(\$s1)\\\\

\problem{}{0}
\solution
\textcolor{blue}{}
a) op = 0, rs = 8, rt = 9, rd = 10, shamt = 0, funct = 34 \\
\begin{tabular}{|c|c|c|c|c|c|}
            \hline
                op&rs&rt&rd&shamt&funct  \\ \hline
                000000&01000&01001&01010&00000&100010 \\ \hline 
            \end{tabular}\\\\
b) op = 0x23, rs = 17, rt = 18, const = 0x4\\
\begin{tabular}{|c|c|c|c|}
            \hline
                op&rs&rt&const  \\ \hline
                100011&10001&10010&0000000000000100 \\ \hline 
            \end{tabular}\

\problem{}{0}
\solution
\textcolor{blue}{}
\$t0 = 0010 0100 1001 0010 0100 1001 0010 0100\\
\$t1 = 0011 1111 1111 1000 0000 0000 0000 0000\\
After comparing bits from most significant bit to least one, we can notice that 4 most significant bit is bigger at \$t1, so \$t1 is bigger.\\
Therefore, slt will set value of \$t2 as 1.\\
Next instruction is beq, and since 1 and 0 are not equal, we go to next instruction, which is to jump to done, ans our final value is 1.
\\\\

\problem{}{0}
\solution
\textcolor{blue}{}
addi \$t0, \$0, 6 \\
sll \$t0, \$t0, 2\\
add \$t0, \$t0, \$s0  - get adress of A[6]\\
lw \$t1, 0(\$t0)  - load value of A[6]\\
add \$s1, \$s1, \$t1\\
sw, \$s1, 0(\$t0)\\\\

\problem{}{0}
\solution
\textcolor{blue}{}
Add 16 most significant bits:\\
0000 0000 0010 0011 = $35_10$\\
lui \$s4, 35\\
Load 16 least significant bits\\
ori \$s4, \$s4, 35\\\\

\problem{}{0}
\solution
\textcolor{blue}{}
\begin{align*}
&\text{addi \$t0, \$0, 0 }\\
&\text{addi \$t1, \$0, 8 }\\
\text{LOOP:}&\text{beq \$t0, \$t1, EXIT} \\
&\text{addi \$s0, \$s0, 4}\\
&\text{addi \$t0, \$t0, 1}\\
&\text{j LOOP}\\
\text{EXIT: }&\\
\end{align*}

\end{document} 